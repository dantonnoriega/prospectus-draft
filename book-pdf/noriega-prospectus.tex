\documentclass[11pt,letterpaper,]{book}
\usepackage{lmodern}
\usepackage{amssymb,amsmath}
\usepackage{ifxetex,ifluatex}
\usepackage{fixltx2e} % provides \textsubscript
\ifnum 0\ifxetex 1\fi\ifluatex 1\fi=0 % if pdftex
  \usepackage[T1]{fontenc}
  \usepackage[utf8]{inputenc}
\else % if luatex or xelatex
  \ifxetex
    \usepackage{mathspec}
  \else
    \usepackage{fontspec}
  \fi
  \defaultfontfeatures{Ligatures=TeX,Scale=MatchLowercase}
  \newcommand{\euro}{€}
    \setmainfont[]{Minion Pro}
    \setsansfont[]{Source Sans Pro}
    \setmonofont[Mapping=tex-ansi]{Consolas}
\fi
% use upquote if available, for straight quotes in verbatim environments
\IfFileExists{upquote.sty}{\usepackage{upquote}}{}
% use microtype if available
\IfFileExists{microtype.sty}{%
\usepackage{microtype}
\UseMicrotypeSet[protrusion]{basicmath} % disable protrusion for tt fonts
}{}
\usepackage[margin=1in]{geometry}
\usepackage{hyperref}
\PassOptionsToPackage{usenames,dvipsnames}{color} % color is loaded by hyperref
\hypersetup{unicode=true,
            pdftitle={Prospectus},
            pdfauthor={Danton Noriega},
            pdfborder={0 0 0},
            breaklinks=true}
\urlstyle{same}  % don't use monospace font for urls
\usepackage{natbib}
\bibliographystyle{apalike}
\usepackage{graphicx,grffile}
\makeatletter
\def\maxwidth{\ifdim\Gin@nat@width>\linewidth\linewidth\else\Gin@nat@width\fi}
\def\maxheight{\ifdim\Gin@nat@height>\textheight\textheight\else\Gin@nat@height\fi}
\makeatother
% Scale images if necessary, so that they will not overflow the page
% margins by default, and it is still possible to overwrite the defaults
% using explicit options in \includegraphics[width, height, ...]{}
\setkeys{Gin}{width=\maxwidth,height=\maxheight,keepaspectratio}
\setlength{\parindent}{0pt}
\setlength{\parskip}{6pt plus 2pt minus 1pt}
\setlength{\emergencystretch}{3em}  % prevent overfull lines
\providecommand{\tightlist}{%
  \setlength{\itemsep}{0pt}\setlength{\parskip}{0pt}}
\setcounter{secnumdepth}{5}

%%% Use protect on footnotes to avoid problems with footnotes in titles
\let\rmarkdownfootnote\footnote%
\def\footnote{\protect\rmarkdownfootnote}

%%% Change title format to be more compact
\usepackage{titling}

% Create subtitle command for use in maketitle
\newcommand{\subtitle}[1]{
  \posttitle{
    \begin{center}\large#1\end{center}
    }
}

\setlength{\droptitle}{-2em}
  \title{Prospectus}
  \pretitle{\vspace{\droptitle}\centering\huge}
  \posttitle{\par}
  \author{Danton Noriega}
  \preauthor{\centering\large\emph}
  \postauthor{\par}
  \predate{\centering\large\emph}
  \postdate{\par}
  \date{April 03 2016}


\usepackage{booktabs}

% Redefines (sub)paragraphs to behave more like sections
\ifx\paragraph\undefined\else
\let\oldparagraph\paragraph
\renewcommand{\paragraph}[1]{\oldparagraph{#1}\mbox{}}
\fi
\ifx\subparagraph\undefined\else
\let\oldsubparagraph\subparagraph
\renewcommand{\subparagraph}[1]{\oldsubparagraph{#1}\mbox{}}
\fi

\begin{document}
\maketitle

{
\setcounter{tocdepth}{1}
\tableofcontents
}
\chapter{Overview}\label{index}

This document contains drafts and ideas for my prospectus.

Chapter 2 is a descriptive paper exploring the policy of improving
health and food equity of SNAP participants through targeted fruit and
vegetables subsidies. The focal point will be on the
\href{http://www.doubleupfoodbucks.org/}{\emph{Double Up Food Bucks}}
(DUFB) program run by the non-profit \emph{Fair Foods Network} based out
of Michigan. A comparison will be made with another subsidy program
called the \emph{Healthy Incentives Pilot} (HIP). I will argue how and
why the DUFB is a more scalable program and better policy if the aim is
increase fruit and vegetable consumption. I will remain neutral on
whether subsidizing fruit and vegetable purchases lead to improved
health outcomes. The paper will conclude with a preview of results from
recently acquired point-of-sale data from a recent pilot of DUFB in 3
independent Michigan grocery stores.

Chapter 3 is a proposed evaluation of the effectiveness of the DUFB
program. ``Effectiveness'' will be defined by the change in total sales
and volume of produce sold within grocery stores that implement DUFB
(treatment group). The control group would be other stores within census
tracks or zip codes where surrounding populations have similar
demographic composition. A difference-in-difference between stores using
DUFB (treatment) and those without (control) will be used to measure the
size of the effect. While an actual experiment is being designed for FFN
by researchers at NYU, the experiment will span through 2016, making
analysis of the data in the short-term difficult. My research will serve
as an initial benchmark to help ground expectations and results---with
full knowledge that results may be biased---before data from the RCT
becomes available.

\section{Ideas}\label{ideas}

\begin{enumerate}
\def\labelenumi{\arabic{enumi}.}
\tightlist
\item
  \textbf{What product purchases correlate (predict) abnormally fast
  increases in the BMI of young children?}

  \begin{itemize}
  \tightlist
  \item
    Clearly, must control for the fact that BMI for kids increases over
    time naturally; expect variation effect will be driven mostly by
    outliers
  \item
    quick and dirty analysis will be to run L1 or L2 penalized
    regression on BMI to see which product purchases seem to predict
    abnormally high BMI increases
  \end{itemize}
\item
  \textbf{What is the redemption rate of DUFB coupons?}

  \begin{itemize}
  \tightlist
  \item
    DUFB coupons are, in effect, free money that can be used on ANY
    produce conditional on the prior purchase of Michigan produce
  \item
    under the assumption that consumers are rational and utility
    maximizing, we would predict a relatively high redemption rate;
    consumers would not leave free money on the table
  \item
    this is unlikely to be the case but the question is, \emph{just how
    high (or low) is the redemption rate?} Do most coupons go
    unredeemed?
  \item
    This will be a technically difficult problem as we do not have panel
    data but the data could still be used to attempt to answer this
    question
  \end{itemize}
\item
  \textbf{What purchasing behavior distinguishes DUFB participants from
  non-DUFB participants?}

  \begin{itemize}
  \tightlist
  \item
    households have already self-selected into DUFB and non-DUFB. do
    their purchasing patterns differ substantially?
  \item
    Is it possible to identify a DUFB purchase if we remove Michigan
    Produce?
  \item
    This will depend a lot on how frequently Michigan produce is
    purchased without DUFB; otherwise, MI produce purchases will be a
    perfect predictor of DUFB transaction
  \end{itemize}
\end{enumerate}

\chapter{Double Up Food Bucks Description}\label{DUFB}

\section{Summary}\label{summary}

\textbf{Problem}

SNAP participants, compared to higher income groups, tend to purchase
more unhealthy foods and fewer fruits and vegetables. This is known as
the ``Nutrition Gap''. These individual choices, in aggregate, produces
a negative externality: poor diet---one consisting of processed foods
and other products high in sugar, salts, and fats---is closely linked to
adverse health outcomes e.g.~obesity, heart disease, and type 2
diabetes. But the health care costs of these diet-related conditions are
not paid for by SNAP participants but by taxpayers. Policymakers,
therefore, should be looking for cost-effective was to incentivize SNAP
participants towards making healthier food choices.

\textbf{Question(s)}

How effective would a targeted subsidy aimed be at reducing the
nutrition gap? Specifically, could a subsidy aimed at increasing the
fruit and vegetables purchases of SNAP participants make a significant
impact on overall diet? And would the costs of the subsidy justify the
health care savings---assuming there are any?

\textbf{Case Study}

A recently successful fruit and vegetable subsidy program for SNAP
participants has been the
\href{http://www.doubleupfoodbucks.org/}{\textbf{Double Up Food Bucks}}
program. The program initially began as way to encourage more SNAP
participants to visits local farmer's markets. DUFB grew dramatically
across Michigan farmer's market from 2009 to 2014. To make the DUFB more
accessible, the USDA provided funding via that 2014 Farm Bill for the
program to be piloted in grocery stores. Four months of point-of-sale
data have collected but \textbf{a rigorous analysis of the point-of-sale
data has yet to be completed}.

\section{Introduction}\label{introduction}

The Supplemental Nutrition Assistance Program (SNAP)---know as ``Food
Stamps'' before to 2008---was created to help mitigate food insecurity
in the US. The intention of policymakers was to ensure a basic, minimal
level of food security for low-income families
\citep{usda_fns_short_2014}. Other than obvious vices, like alcohol and
cigarettes, there are no restrictions on what food items (excluding pet
food) can be purchased with SNAP. Consumer are free to purchase
unhealthy and/or healthy food. But in the last few years, these light
restrictions on what can and cannot be purchases with SNAP have fallen
under scrutiny.

\subsection{Purchasing Patterns of SNAP
Beneficiaries}\label{purchasing-patterns-of-snap-beneficiaries}

Research has found that SNAP participants, at best, consume same amount
of unhealthy foods (e.g.~sugar-sweetened beverages, baked goods, snacks,
candy) compared to non-participants and ineligible households
\citep{todd_caloric_2014, hoynes_snap_2014}. This is not a positive
result. The average US household, regardless of income, fails to meet
the USDA's dietary guidelines \citep{usda_scientific_2015}. Most US
households purchase and consume far too much meat and foods rich in
sugars and fats, and far too few fruits, vegetables and whole grains
\citep{frazao_americas_1999}. Far more research has found that SNAP
participants are significantly \emph{less} likely to meet USDA dietary
guidelines than the average US household and much \emph{more} likely to
consume unhealthy foods
\citep{andreyeva_dietary_2015, leung_dietary_2012, nguyen_supplemental_2015, wolfson_fruit_2015}.

\subsection{Poor Diet and Metabolic Risk
Factors}\label{poor-diet-and-metabolic-risk-factors}

Frequent and continuous consumption of unhealthy foods---as is captured
by ``unhealthy'' part of ``unhealthy foods''---is worrisome to health
experts and policy makers. There is a direct and well-established link
between poor diet, metabolic risk factors (e.g.~obesity, high blood
pressure, type 2 diabetes, raised blood lipids), and cardiovascular
disease and strokes. Metabolic risk factors (MRFs) lead to deaths that
are considered ``avoidable'' by the World Health Organization and the
Centers for Disease Control and Prevention
\citep{who_global_2014, cdc_vital_2013}. Governments and individuals,
therefore, have a role in reducing the incidence rates of MRFs because
both can help reduce the consumption of unhealthy foods and increase the
consumption ones. The US, in particular, given it has the highest per
capita health care costs in the world, could greatly benefit
economically from policy interventions aimed at promoting better diet
quality and reducing the rate (and cost) of MRFs.

\subsection{The Economic Costs of Diet-Related Chronic
Diseases}\label{the-economic-costs-of-diet-related-chronic-diseases}

Cardiovascular disease, stroke, diabetes, and certain cancers can be
considered ``diet-related'' chronic diseases if the person suffering
from the disease exhibits any of the MRFs and consumes an unhealthy
diet. While it is possible to eat well and still suffer from a heart
attack or a stroke, the likelihood is relatively low compared to a
counter-factual self that eats poorly. On the other hand, it is a well
established that a poor diet greatly increases the risk of any of the
MRFs, which in turn greatly increases the risk of chronic diseases
\citep{burke_gl_impact_2008}.

{[}\textbf{LIST OF THE COSTS OF CHRONIC DISEASES AFTER HAVING
ESTABLISHED POOR DIET-\textgreater{}MRF LINK}{]}

\subsection{Adjustments to SNAP}\label{adjustments-to-snap}

There is mixed evidence about whether participating in SNAP leads to
increases in MRFs, particularly obesity. A review of the literature and
empirics by \citet{meyerhoefer_relationship_2011} highlights how poor
methodology has yielded a confusing landscape of results. But while no
definitive link between SNAP participation and poor health has not been
made, it is still a fact that the average American---SNAP participant or
not---has a poor diet.

There has been enough evidence of SNAP participants using their benefits
to purchase unhealthy foods that some policy makers have asked why
restrictions on some food items (e.g.~sugar-sweetened beverages) have
yet to be imposed \citep{barnhill_impact_2011}.

{[}\textbf{MORE INFO ON SNAP RESTRICTIONS}{]}

Within the context of US obesity epidemic, alarmingly high rates of
MRFs, and accelerating government health care cost, some policy makers
have suggested making SNAP more like the Woman, Infants, and Children
(WIC) program. The WIC program provides food vouchers which limit
participants to a select group of products. These products are
specifically selected to be healthy to ensure women and their children
receive nutritious, healthy foods. This is in contrast to SNAP, where
there are few restrictions and participants are free to purchase
cookies, chips, soda, and any other assortment of process foods that
have no WIC voucher equivalent.

The purchase of unhealthy foods by SNAP participants is considered
problematic because SNAP participants also tend to receive some form of
government subsidized health care (e.g.~Medicaid). The government is
therefore providing a benefit transfer to individuals who generally tend
to spend those benefits on a largely unhealthy diets, which are closely
linked to higher morbidity rates. But the financial cost of treating
MRFs are not borne by the participants of government subsidized health
care. The government is therefore helping fund, and then pay for, a
negative externality produced by the individual consumption choices of
most SNAP participants. Ideally, given the strong link between diet and
morbidity, the government would prefer SNAP participants make, on the
whole, healthier food choices when using SNAP benefits
\citep{richards_rewarding_2013, brownell_kd_supplemental_2011, guthrie_usda_2007}.
It would lower health care costs and save the government money.

These unhealthy food choices, some have argued, are a function of
limited \emph{budget constraints} \citep{andreyeva_impact_2010}. That
is, folks purchase unhealthy foods because they are cheaper, per
calorie, then healthy foods, and make it easier to satisfy food
requirements given a limited budget. Therefore, while not optimal for
health, it is optimal/rational given a small food budget. But SNAP
participants, like any and all other consumers, do not entirely
optimizing on price. All consumers optimize across many dimensions
(e.g.~habit, taste, price, convenience etc). It is not entirely clear
that if SNAP participants did receive more money that they would
necessarily eat better \citep{an_effectiveness_2013}.

If SNAP participants desire fruits and vegetables but they lie beyond
their limited means, then increasing their purchasing power
\emph{overall} would be sufficient to make fruits and vegetables more
accessible. This would imply that the most ``straightforward''
solution---if one ignores the political challenges---would be to
increase benefit transfers to SNAP participants. This of course is not
what happens when low-income households receive marginal boosts of
income; a dislike for fruits and vegetables is endemic across all income
groups in the United States \citep{frazao_americas_1999}. While
proclivity to purchase for fruits and veggies does rise with income and
education, the increases are not dramatic. SNAP participants, when they
to receive a marginal increase in SNAP benefits, do not suddenly begin
purchasing more fruits and vegetables \citep{hayden_income_2003}.
Concern for eating healthy is a luxury; it becomes a priority only after
incomes rise enough to alleviate more pressing concerns---like hunger or
buying enough food to last until the next benefit transfer.

{[}\textbf{ADD MORE ABOUT HOW TARGET SUBSIDIES LIKE DUFB CAN PROVIDE A
BETTER INCENTIVE FOR HOUSEHOLDS TO PURCHASE MORE FRUITS \& VEGGIES
PROVIDE}{]}

\section{The Double Up Food Bucks
Program}\label{the-double-up-food-bucks-program}

{[}\textbf{Describe DUFB}{]}

\section{Data and Data Collection
Process}\label{data-and-data-collection-process}

These data were collected from 3 independent grocery stores in the
Detroit area. After pre-processing, the combined dataset contains 6.5
million point-of-sale transactions. Observed are item name, item
department, UPC, price paid, tender (e.g. \texttt{cash},
\texttt{debit}), and the last 4 digits of any card used as tender. Apart
from the last 4 digits of a debit, credit, or EBT card, there is no way
to uniquely identifying information about consumers. None of the
participating grocery stores had existing loyalty card programs.

Double Up Food Bucks are observed as items purchased for zero dollars.

\section{Preview of Results}\label{preview-of-results}

\section{Misc.}\label{misc.}

\subsection{Why is this interesting and why should we
care?}\label{why-is-this-interesting-and-why-should-we-care}

\begin{itemize}
\tightlist
\item
  As a Society

  \begin{itemize}
  \tightlist
  \item
    There are societal costs (externalities) to having SNAP participants
    eat unhealthy foods.

    \begin{itemize}
    \tightlist
    \item
      Increases the costs of Medicaid and Medicare (more heart disease,
      type 2 diabetes)
    \item
      Reduces productivity (poor diet, poor health results in worse
      labor market participation)
    \end{itemize}
  \end{itemize}
\item
  As policymakers

  \begin{itemize}
  \tightlist
  \item
    Can SNAP be tweaked to nudge participants towards healthier foods
    without restricting food options?
  \item
    If so, can it be done such that the costs are justified by future
    savings?
  \end{itemize}
\end{itemize}

\subsection{What have others done to solve the problem or answer the
question?}\label{what-have-others-done-to-solve-the-problem-or-answer-the-question}

\begin{itemize}
\tightlist
\item
  Health Incentives Pilot (HIP)

  \begin{itemize}
  \tightlist
  \item
    Large, expensive RCT designed by USDA FNS run in Massachusetts
  \item
    RCT and evaluation was conducted by Abt Associates Inc.
  \item
    Complicated with very little (\emph{\%6}) of the overall costs going
    to the actual subsidy.
  \item
    Evidence that the program was not well understood by participants.
  \item
    Results were an small increase in serving size of fruits/vegetables
    per month

    \begin{itemize}
    \tightlist
    \item
      This is not a surprise; one should expect the consumption of any
      good that becomes cheaper to go up.
    \end{itemize}
  \item
    Problems

    \begin{itemize}
    \tightlist
    \item
      Survey-based data
    \item
      POS data did exists but did not seem to be used.
    \item
      No real understanding of how other purchases were affected (did
      households substitute healthy for unhealthy or just purchase more
      food overall?)
    \end{itemize}
  \end{itemize}
\item
  Double Up Food Bucks

  \begin{itemize}
  \tightlist
  \item
    Program started by the Fair Food Network in 2009 to increase
    accessibility of Farmer's Markets to low-income communities in
    Detroit, MI area.

    \begin{itemize}
    \tightlist
    \item
      Provided a dollar-for-dollar match of SNAP benefits (capped at
      \$20)
    \item
      Participants received a token that could then be used on any
      locally grown (MI) produce.
    \end{itemize}
  \item
    Popular with participants, community, and with local farmers.
  \item
    Considered successful and received money from US Gov to expand into
    grocery stores
  \item
    Problems

    \begin{itemize}
    \tightlist
    \item
      Severe self-selection issues (folks visiting farmer's market are
      different from rest of SNAP population)
    \item
      No way to gather transaction-level data about what was being
      purchased
    \end{itemize}
  \item
    \textbf{Has no been evaluated to see if actually produces a
    worthwhile treatment effect}

    \begin{itemize}
    \tightlist
    \item
      That said, should that matter? Perhaps given money to poor people
      that then goes to local farmers is a good use of public funds.
    \end{itemize}
  \end{itemize}
\item
  Maybe talk about the policy to change the timing of SNAP
  disbursements?
\end{itemize}

\subsection{What have I done to answer the
question?}\label{what-have-i-done-to-answer-the-question}

\begin{itemize}
\tightlist
\item
  Received data from FFN and started cleaning it

  \begin{itemize}
  \tightlist
  \item
    Data is from 3 independent grocery stores that have implement DUFB
    and gathered POS data
  \end{itemize}
\item
  Drafted report with list of questions about how the process works
\item
  Phone call with FFN to answer questions
\item
  Working on follow-up report
\item
  Problems

  \begin{itemize}
  \tightlist
  \item
    Do no have access to data prior to DUFB implementation
  \item
    Not panel data; cannot uniquely identify individuals across time
  \item
    need to find a way to distinguish sub transactions
  \end{itemize}
\end{itemize}

\bibliography{packages.bib,book.bib}

\end{document}
