\documentclass[12pt,letterpaperpaper,]{book}
\usepackage{lmodern}
\usepackage{amssymb,amsmath}
\usepackage{ifxetex,ifluatex}
\usepackage{fixltx2e} % provides \textsubscript
\ifnum 0\ifxetex 1\fi\ifluatex 1\fi=0 % if pdftex
  \usepackage[T1]{fontenc}
  \usepackage[utf8]{inputenc}
\else % if luatex or xelatex
  \ifxetex
    \usepackage{mathspec}
  \else
    \usepackage{fontspec}
  \fi
  \defaultfontfeatures{Ligatures=TeX,Scale=MatchLowercase}
\fi
% use upquote if available, for straight quotes in verbatim environments
\IfFileExists{upquote.sty}{\usepackage{upquote}}{}
% use microtype if available
\IfFileExists{microtype.sty}{%
\usepackage{microtype}
\UseMicrotypeSet[protrusion]{basicmath} % disable protrusion for tt fonts
}{}
\usepackage[margin=1in]{geometry}
\usepackage{hyperref}
\hypersetup{unicode=true,
            pdftitle={DRAFT Dissertation Prospectus},
            pdfauthor={Danton Noriega},
            pdfborder={0 0 0},
            breaklinks=true}
\urlstyle{same}  % don't use monospace font for urls
\usepackage{natbib}
\bibliographystyle{apalike}
\usepackage{longtable,booktabs}
\usepackage{graphicx,grffile}
\makeatletter
\def\maxwidth{\ifdim\Gin@nat@width>\linewidth\linewidth\else\Gin@nat@width\fi}
\def\maxheight{\ifdim\Gin@nat@height>\textheight\textheight\else\Gin@nat@height\fi}
\makeatother
% Scale images if necessary, so that they will not overflow the page
% margins by default, and it is still possible to overwrite the defaults
% using explicit options in \includegraphics[width, height, ...]{}
\setkeys{Gin}{width=\maxwidth,height=\maxheight,keepaspectratio}
\IfFileExists{parskip.sty}{%
\usepackage{parskip}
}{% else
\setlength{\parindent}{0pt}
\setlength{\parskip}{6pt plus 2pt minus 1pt}
}
\setlength{\emergencystretch}{3em}  % prevent overfull lines
\providecommand{\tightlist}{%
  \setlength{\itemsep}{0pt}\setlength{\parskip}{0pt}}
\setcounter{secnumdepth}{5}
% Redefines (sub)paragraphs to behave more like sections
\ifx\paragraph\undefined\else
\let\oldparagraph\paragraph
\renewcommand{\paragraph}[1]{\oldparagraph{#1}\mbox{}}
\fi
\ifx\subparagraph\undefined\else
\let\oldsubparagraph\subparagraph
\renewcommand{\subparagraph}[1]{\oldsubparagraph{#1}\mbox{}}
\fi
\usepackage{booktabs}
\usepackage{lscape}
\usepackage{longtable}
\usepackage{bm}
\usepackage{isomath}
\usepackage{setspace}
\usepackage[bf,singlelinecheck=off]{caption}
\newcommand{\matr}[1]{\mathbf{#1}}
\newcommand{\blandscape}{\begin{landscape}}
\newcommand{\elandscape}{\end{landscape}}

\setromanfont[Mapping=tex-text]{Minion Pro}
\setsansfont[Mapping=tex-text]{Source Sans Pro}
\setmonofont[Mapping=tex-text,Scale=0.8]{Source Code Pro}

\usepackage{framed,color}
\definecolor{shadecolor}{RGB}{248,248,248}

\renewcommand{\textfraction}{0.05}
\renewcommand{\topfraction}{0.8}
\renewcommand{\bottomfraction}{0.8}
\renewcommand{\floatpagefraction}{0.75}

\renewenvironment{quote}{\begin{VF}}{\end{VF}}
\let\oldhref\href
\renewcommand{\href}[2]{#2\footnote{\url{#1}}}

\ifxetex
  \usepackage{letltxmacro}
  \setlength{\XeTeXLinkMargin}{1pt}
  \LetLtxMacro\SavedIncludeGraphics\includegraphics
  \def\includegraphics#1#{% #1 catches optional stuff (star/opt. arg.)
    \IncludeGraphicsAux{#1}%
  }%
  \newcommand*{\IncludeGraphicsAux}[2]{%
    \XeTeXLinkBox{%
      \SavedIncludeGraphics#1{#2}%
    }%
  }%
\fi

\makeatletter
\newenvironment{kframe}{%
\medskip{}
\setlength{\fboxsep}{.8em}
 \def\at@end@of@kframe{}%
 \ifinner\ifhmode%
  \def\at@end@of@kframe{\end{minipage}}%
  \begin{minipage}{\columnwidth}%
 \fi\fi%
 \def\FrameCommand##1{\hskip\@totalleftmargin \hskip-\fboxsep
 \colorbox{shadecolor}{##1}\hskip-\fboxsep
     % There is no \\@totalrightmargin, so:
     \hskip-\linewidth \hskip-\@totalleftmargin \hskip\columnwidth}%
 \MakeFramed {\advance\hsize-\width
   \@totalleftmargin\z@ \linewidth\hsize
   \@setminipage}}%
 {\par\unskip\endMakeFramed%
 \at@end@of@kframe}
\makeatother

\newenvironment{rmdblock}[1]
  {
  \begin{itemize}
  \renewcommand{\labelitemi}{
    \raisebox{-.7\height}[0pt][0pt]{
      {\setkeys{Gin}{width=3em,keepaspectratio}\includegraphics{images/#1}}
    }
  }
  \setlength{\fboxsep}{1em}
  \begin{kframe}
  \item
  }
  {
  \end{kframe}
  \end{itemize}
  }
\newenvironment{rmdnote}
  {\begin{rmdblock}{note}}
  {\end{rmdblock}}
\newenvironment{rmdcaution}
  {\begin{rmdblock}{caution}}
  {\end{rmdblock}}
\newenvironment{rmdimportant}
  {\begin{rmdblock}{important}}
  {\end{rmdblock}}
\newenvironment{rmdtip}
  {\begin{rmdblock}{tip}}
  {\end{rmdblock}}
\newenvironment{rmdwarning}
  {\begin{rmdblock}{warning}}
  {\end{rmdblock}}

\usepackage{makeidx}
\makeindex

\urlstyle{tt}

\usepackage{amsthm}
\makeatletter
\def\thm@space@setup{%
  \thm@preskip=8pt plus 2pt minus 4pt
  \thm@postskip=\thm@preskip
}
\makeatother

\title{DRAFT Dissertation Prospectus}
\author{Danton Noriega}
\date{November 22, 2016}

\begin{document}
\maketitle

\setlength{\abovedisplayskip}{-5pt}
\setlength{\abovedisplayshortskip}{-5pt}
\mainmatter

{
\setcounter{tocdepth}{2}
\tableofcontents
}
\chapter*{Overview}\label{overview}
\addcontentsline{toc}{chapter}{Overview}

Chapter 1 is an evaluation of the effectiveness of the Double Up Food
Bucks program. ``Effectiveness'' will be defined by the change in total
sales and volume of produce sold within a subset of grocery stores that
implement Double Up (treatment group). The control group comprises 15
stores where Double Up was not implemented. A regression discontinuity
design and difference-in-differences, between stores using Double Up
(treatment) and those without (control), will be used to measure the
size of the effect.

Improving health and food equity of SNAP participants is the broader
policy concern. The mechanism is a financial incentive---Double Up Food
Bucks---designed to increase fruit and vegetables consumption. A
comparison will be made with another financial incentive program called
the \emph{Healthy Incentives Pilot} (HIP). I will argue how and why an
evaluation of the Double Up program is an important addition to the
current literature.

\chapter{An Evaluation of the Double Up Food Bucks
program}\label{chapter-1}

\section*{Introduction}\label{intro-1}
\addcontentsline{toc}{section}{Introduction}

Chronic conditions like obesity, heart disease, and other metabolic risk
factors (stroke, type II diabetes, etc.) are estimated to cost the US
health care system between 200 to 400 billion dollars annually
\citep{cawley_medical_2012, chatterjee_checkup_2014}. More importantly,
these diseases account for hundreds of thousands of deaths each year.
Heart disease alone is the leading cause of death for all persons in the
US, with stroke fifth and diabetes seventh
\citep{national_center_for_health_statistics_health_2015}. Diet is
closely linked to these conditions, particularly obesity and
cardiovascular disease. There is strong evidence that a diet high in (1)
vegetables, fruits, nuts, unsaturated oils, fish, and poultry, but low
in (2) red and processed meat and sugar-sweetened foods and drinks,
helps lower body weight, blood pressure, and the risk of cardiovascular
disease
\citep{mente_systematic_2009, nutrition_evidence_library_series_2014}.
Improving the diet of Americans has therefore become an increasing
priority for the United States, especially for struggling families that
participate in the Supplemental Nutrition Assistance (SNAP) program.

SNAP is a federal aid program administered by the Food and Nutrition
Service (FNS), an agency of the U.S. Department of Agriculture (USDA).
At 74 billion dollars in FY2015 with roughly 45.8 million participants,
it is the largest food assistance program in the US
\citep{usda_fns_supplemental_2016}. To be eligible for SNAP, a household
must be sufficiently budget constrained that hunger is considered likely
without assistance. Eligibility is a function of countable resources,
vehicle ownership and value, household size, gross or net monthly
income, household composition, and meeting certain work
requirements.\footnote{For more details, visit
  \url{http://www.fns.usda.gov/snap/eligibility}} Some eligibility
requirements vary by state, but in general, a family with less than
\$2000 in countable resources, where the adults work at least part-time
earning a gross (net) monthly income at or below 130\% (100\%) of the
federal poverty line, is eligible to receive SNAP benefits. Aside from a
few restrictions---no alcohol, tobacco, non-food items, read-to-eat
meals, or hot foods---households can use SNAP benefits to purchase any
foods that will be prepared and consumed at home. Unfortunately, the
purchasing patterns of the average SNAP household are not conducive to a
healthy diet.

Research on the dietary patterns of households receiving SNAP benefits
has found that they are significantly \emph{less} likely to meet USDA
dietary guidelines than the average US household and much \emph{more}
likely to consume unhealthy foods
\citep{andreyeva_dietary_2015, nguyen_supplemental_2015, wolfson_fruit_2015}.
A smaller set of research has found that SNAP households, at best,
consume same amount of unhealthy foods (e.g.~sugar-sweetened beverages,
baked goods, snacks, candy, etc) compared to SNAP-ineligible households
\citep{todd_caloric_2014, hoynes_snap_2015}. In other words, SNAP
households consume foods that are less healthy or about the same as
SNAP-ineligible households. This is a concerning result given that most
US households, regardless of income, already purchase and consume far
too much meat and foods rich in sugars and fats, and far too few fruits,
vegetables and whole grains
\citep{usda_scientific_2015, frazao_high_1999}. However, the purpose of
SNAP is to keep struggling families from going hungry, not to ensure
they consume the best possible diet. SNAP is designed to act like cash,
helping families access more food than they could otherwise do so
without assistance \citep{hoynes_snap_2015}. It is therefore not a
failing of the SNAP program if benefits are used to purchase unhealthy
foods.

The SNAP program could be change such that it could continue satisfying
its role as an anti-hunger program while simultaneously encouraging
healthier purchases. \citet{blumenthal_strategies_2014} and
\citet{leung_qualitative_2013} both surveyed a field of stakeholders and
policy experts in the SNAP program about what they would do to improve
the dietary quality of purchases. In both studies, restricting the
purchase of unhealthy foods (e.g.~sugar-sweetened beverages) and
promoting healthy purchases through monetary incentives were the two
most popular improvements (i.e.~ranked the highest or most often
suggested).\footnote{It should be mentioned that there were other, less
  popular recommendations, such as modifying how SNAP benefits are
  distributed and improving nutrition education. For more details, see
  \citet{blumenthal_strategies_2014} and \citet{leung_qualitative_2013}}.

One common suggestion is to restrict the SNAP program to the same set of
eligible foods as the Woman, Infants, and Children (WIC) program
\citep{dinour_food_2007}. The WIC program provides food vouchers which
limit households to a select group of food products. These food products
are specifically selected to ensure women and their children receive
nutritious, healthy foods. In other words, the WIC program, by design,
places restrictions on food choices by defining a list of
\emph{eligible} items, as opposed to the SNAP program, which defines a
list of \emph{ineligible} items. Another common, and simpler, suggestion
is to expand the existing list of ineligible items (e.g.~alcohol) with
products that are unambiguously lacking in nutrition and easy to
identify, like soda or candy. New York City, for example, attempted to
ban the purchase of sugar-sweetened beverages, and the state of Maine
attempted to restrict the purchase of sodas, candy, and any other
taxable food items \citep{gundersen_snap_2015}. Both restrictions were
overturned by the USDA.

There are problems with ``improving'' the SNAP program by implementing
even greater purchasing restrictions. First, there is no reason to
believe that such a restriction would work. The restriction assumes
that, under WIC-like requirements, households will substitute healthy
foods for unhealthy foods when using SNAP benefits. What would most
likely happen is that households would shift to purchasing unhealthy
foods with cash. Second, such restrictions would likely lead to a drop
in SNAP participation \citep{gundersen_snap_2015}. Restricting choice is
a paternalistic policy that would further stigmatize SNAP participation.
It would give the impression that SNAP beneficiaries are assumed to have
worse diets and that they cannot be trusted to make healthy food
purchases. Participation would also drop due to increased transactions
costs of purchasing items with SNAP. Not all stores would clearly mark
which items were SNAP eligible nor should participants be expected to
remember. The result would be longer, more frustrating shopping trips.
Lastly, it is important to remember that for many SNAP recipients,
freedom of choice is what makes the SNAP program popular and easy to use
\citep{edin_snap_2013}.

The most popular ``improvement'' was providing a monetary incentive to
SNAP participants for purchasing healthy foods
\citep{blumenthal_strategies_2014, leung_qualitative_2013}. Monetary (or
financial) incentives, in this context, tend to be a rebate or voucher
awarded to SNAP households for using their benefits to buy certain
healthful foods, generally mineral-rich and nutrient-dense fruits and
vegetables (i.e.~leafy greens but not white potatoes). These monetary
incentives for buying ``targeted'' fruits and vegetables (aka TFVs) are
exclusive to SNAP participants. Much like a grocery stores loyalty card
or a student ID card, retailers can ``discriminate on price'' (aka
``target the incentive'') using SNAP Electronic Benefit Transfer (EBT)
cards to identifying eligible participants. Monetary incentives in the
food retail environment are popular for two main reason. First, the
framing of the ``improvement'' is positive. Instead of ``punishing''
SNAP participants through paternal restriction or disincentives (not
covered), monetary incentives reward participants for healthy shopping
behavior \citep{gundersen_snap_2015}. Retailers also prefer the positive
framing of monetary incentives. For the moment, monetary incentives
programs for SNAP participants are not wide spread. Therefore, taking up
an incentive program, assuming the cost of implementation isn't too
expensive, creates an opportunity for retailers to differentiate
themselves from their competitors \citep{hartmann_corporate_2011}. The
second reason is a strong theoretical framework established by
neoclassical economics supporting incentives as an effective mechanism
for changing human behavior. In practice, however, incentives have had
mixed results, but there is building evidence that incentives may work
in the food retail space.

How, why, and to what effect incentives may encourage SNAP participants
to purchase more targeted fruits and vegetables will be discussed in
detail below, and is the motivating question behind this paper.

\subsection*{Financial Incentives to Encourage Healthy Food
Purchases}\label{financial-incentives-to-encourage-healthy-food-purchases}
\addcontentsline{toc}{subsection}{Financial Incentives to Encourage
Healthy Food Purchases}

Encouraging healthy behavior through financial incentives has a long
history. Results are mixed. For example, financial incentives have been
shown to help individuals commit to regular exercise, improve dieting,
increase weight loss, and to quit smoking, but the intended effect of
the financial incentives were often only short-term (see
\citet{gneezy_when_2011} and \citet{cawley_economy_2015} for an
overview). \citet{gneezy_when_2011} also explain, through a review of
the literature, that depending on the context and design, incentives can
backfire, producing an effect known as ``crowd out''. Crowd out occurs
when an incentive displaces the intrinsic reward of a behavior
(originally defined for ``prosocial'' behavior, like volunteering or
giving blood; see \citet{benabou_incentives_2006}). The behavior then
becomes dependent on the extrinsic reward. As a result, having shifted
from being intrinsically rewarding to extrinsically rewarding, the
positive behavior continues only as long as the monetary incentive is
provided. More significantly, the intrinsic reward of the behavior does
not return once it has been ``crowded out''. Therefore, the long-term
effect of a monetary incentive can be negative, despite showing positive
effects in the short-term. That said, incentives can produce successful
long-term results if they are instead used as a mechanism to build good
habits. This requires that the incentives be salient and produce
immediate feedback without neglecting behavioral findings such as loss
aversion and mental accounting \citep{john_financial_2011}.

Given the research on incentives, it is reasonable to assume that a
monetary incentive for SNAP participants to purchase healthy foods, like
fruits and vegetables, may fail or even backfire. Should the act of
purchasing healthy foods be intrinsically rewarding to SNAP shoppers,
introducing an incentive may produce a ``crowd out'' effect. However,
recent field experiments find that incentives can establish healthy food
choice as a habit, possibly overriding any crowd out. Daily incentives
encourage children to make healthier food choices in school lunchrooms,
who in turn develop positive, long-term food habits
\citep{loewenstein_habit_2016, list_behavioralist_2015, belot_incentives_2014}.
Outside of the school lunchroom environment,
\citet{list_incentives_2015} found that similar habit formation is
possible with incentives in a more traditional food retail environment.
In their experiment, \citet{list_incentives_2015} provided an incentive
to 222 shoppers (\$1) to use their rewards cards, but then randomly
assigned each participant to the control group to one of three
interventions: \emph{information}, \emph{incentive}, or
\emph{combination}. The \emph{information} treatment was a flyer with
tips on how to prepare fruit and vegetable dishes as well as the health
benefits of eating more fruits and veggies. The \emph{incentive} was an
additional dollar for every 5 cups of targeted fruits and vegetables
(TFVs) purchased. The \emph{combination} treatment group included both.
The intervention lasted for 5 months but each group continued to be
observed for roughly 6 weeks months after. The \emph{information}
intervention had no effect, but the \emph{incentive} and
\emph{combination} interventions on average doubled their purchase of
fruits and vegetables in comparison to the control group. Most
promisingly, the gap persisted with minimal shrinkage for 6 weeks
following the end of the intervention. However, there was no follow up
after the 6-week post-intervention period. It is therefore possible the
gap closed over multiple months (as opposed to multiple weeks).

The design, food environment, and target of the incentive in each of
these experiments is important. First, the incentive in these
experiments is designed to be salient and immediate. In the school
lunchroom experiments, the children are aware of the incentive and
receive the reward (e.g.~a small token) immediately after selecting the
healthy food item. Likewise, the shoppers received their additional \$1
reward for every 5 cups of TFVs at checkout. One distinction between the
designs is frequency. The children are exposed to the incentive every
school day in the lunchroom experiments. The shoppers, on the other
hand, were exposed as frequently or as infrequently as they chose. This
distinction is important, as the latter better reflects the experience
of shoppers using the SNAP benefits. Second, the food environment is
important because it determines what choices are available. The children
have a finite set of options in the school lunchroom and they also have
no outside option (besides not eating lunch). The children optimize on a
relatively small set of choices and, for the duration of the
intervention, the incentive always existed. Food retail environments are
drastically different. There are numerous competing food choices and
generally other outside options.\footnote{It should be noted that this
  is not always the case. In \citet{list_incentives_2015}, for example,
  the store that was selected for the experiment was one of the few
  places local shoppers could find fresh produce. The store itself was
  located in one of Chicago's poorer neighborhoods.} It is substantially
more difficult for the shopper to optimize over such a large set of
choices. Last, and most obviously, one set of studies targets children,
the other adults. A priori, we would expect a monetary incentive to
affect children differently than adults. The fact that habit formation
through incentives appears possible for both target groups is promising.

Research where SNAP participants are the target group is nascent. The
USDA's Food and Nutrition Services (FNS) ran the first large scale
randomized control trial investigating the impact of a financial
incentive for targeted fruits and vegetables in 2011. The experiment was
called the Healthy Incentives Pilot (HIP). HIP is the precursor to every
incentive program currently being funding by the USDA. It also provides
the data for the few papers recently published on incentives for SNAP
participants.

\subsection*{The Healthy Incentives
Pilot}\label{the-healthy-incentives-pilot}
\addcontentsline{toc}{subsection}{The Healthy Incentives Pilot}

A brief overview of the Healthy Incentives Pilot is necessary to provide
context to, and contrast with, more recent financial incentive programs.

The UDSA's Food and Nutrition Services designed the Healthy Incentives
Pilot. The pilot was funded by the Food, Conservation, and Energy Act of
2008 to test whether financial incentives would increase consumption of
targeted fruits and vegetables (TFVs). SNAP participants were the target
group.

HIP was designed as a large scale randomized control trial (RCT). FNS
partnered with the Massachusetts Department of Transitional Assistance
to implement HIP. The pilot lasted from early 2011 to the end of 2012.
The population included all 55,095 SNAP participants in Hampden County,
MA. Hampden County is the poorest county in Massachusetts and has the
highest rates of obesity and other diet-related chronic illness
(e.g.~type 2 diabetes).

Of the 55,095 SNAP participants, 7,500 were randomly assigned to the
treatment group. The remainder fell into the control group. The
treatment was a 30 cent (or 30\%) rebate on every dollar spent on TFVs.
The rebate was capped at \$60 per month. To receive the rebate, selected
SNAP participants had to use their EBT cards at participating retailers.
The rebate, which was returned to their EBT account, could then be used
on any food item. That is, the rebate could only be earned buying TFVs,
but could be redeemed buying any SNAP eligible food item. Most HIP
participants spent about \$12 a month on TFVs, earning an average of
\$3.65 per month in rebates---drastically lower than the \$60 per month
rebate cap.

The evaluation was conducted using 24-hour dietary recall surveys. A
total of 5,000 participants were selected to be surveyed, even split
between treatment and control (2,500 HIP, 2,500 non-HIP). The first
survey was conducted prior to the start of the pilot. This established a
baseline. The second survey occurred 4 to 6 months in to the pilot and
the third survey occurred 9 to 11 months in. (The variation, e.g.~4 to 6
months, was due to the treatment being implemented in 3 waves of 2500.)

The evaluation found that the 30\% rebate lead to about a 26\% increase
in consumption of TFVs. This was equivalent to about 0.24 cups of TFVs.
Roughly 60\% of the increase was due to increased vegetable consumption
and 40\% due to increased fruit consumption. The effect, in absolute
terms (0.24 cups), seems small. But a \(0.87\) price elasticity,
relative to other results in the literature, is quite high---\(0.7\) and
\(0.48\), on average, for fruits and vegetables, respectively
\citep{andreyeva_impact_2010}.

Despite some limitations and technical problems---underreporting on the
24-hour recall survey and system glitches early in the pilot (see pages
60 and 208-210 of \citet{bartlett_evaluation_2014})---HIP was considered
to be an overall success
\citep{klerman_short-run_2014, olsho_financial_2016}. It implemented on
of the largest, most complex RCTs to isolate how incentives can increase
household consumption of TFVs. It also provided a feasible model for
nationwide expansion (assuming cost reductions due to economies of
scale; see \citet{an_nationwide_2015}).

HIP also provides a framework for understanding how a financial
incentive, expanded dramatically in one geographic area, could improve
TFV consumption. But, as noted in the final HIP report, one of the most
prominent retailers in Hampden County chose not to participate (page 61,
\citet{bartlett_evaluation_2014}). Its third-party processor decided it
was too difficult and too costly to implement the financial incentive on
its point-of-sale technology. This strategic behavior by the retailer,
which had a significant presence in Hampden County, impacted where
participants could use the incentive.

Most financial incentive programs work at the local level, expanding
non-randomly. We should anticipate certain retailers (firms) to behave
strategically when participating in any of these incentive programs.
Likewise, we should anticipate voluntary (non-random) self-selection by
SNAP beneficiaries into these financial incentives programs. To this
end, more research is needed to understand the impact of incentive
programs under \emph{real-world} conditions. HIP provided evidence that
incentive programs can work, but barring state-wide or nation-wide
adoption of point-of-sale financial incentives, we should expect growth
to occur organically under non-experimental conditions.

An example of such a financial incentive program for SNAP participants
is the Double Up Food Bucks program (DUFB or Double Up). The non-random
expansion and impact of this financial incentives program will remain
the focus of this paper

\subsection*{The Double Up Food Bucks
Program}\label{the-double-up-food-bucks-program}
\addcontentsline{toc}{subsection}{The Double Up Food Bucks Program}

The success of HIP paved the way for the Food Insecurity Nutrition
Initiative (FINI), established by section 4208(b) of the Agricultural
Act of 2014 (aka 2014 Farm Bill). FINI---a 100-million-dollar
initiative---in turn piloted numerous non-profit financial incentive
programs aimed at improving the diets of SNAP participants.

Of specific interest is Double Up Food Bucks (DUFB or Double Up), an
incentives-based program funded by FINI. In 2009, the non-profit
organization Fair Food Network (FFN) launched the Double Up Food Bucks
program in Detroit, Michigan. The intention of the program was to get
more low-income families visiting and participating in local Detroit
farmer's markets. The mechanism for increasing participation was a
dollar-for-dollar match of locally grown fruit and vegetable purchases.
This subsidy was accessible only to low-income families receiving SNAP
benefits, who could exchange up to \$20 of their benefits for a wooden
token that could be used on up to \$40 worth of locally grown produce.

The DUFB program was considered successful given it had expanded to more
than 150 farmer's markets in 2014 from just 5 farmer's markets in 2009.
SNAP benefits have been used more than 200,000 times to purchase fresh
produce, with more than 10,000 first time SNAP customers visiting
farmer's markets in 2013 alone \citep{fair_food_network_double_2014}.
The program is considered by Fair Food Network to be a ``three-fold''
win given that the program helps local low-income families buy more
fresh produce, provides new customers for local farmer's, and stimulates
the local food economy. Relative to farmer's markets in other states,
DUFB did seem to be bringing in substantially more SNAP dollars (\$1.7
million in Michigan versus \$307,000 in Illinois, the second largest).

A 5.17 million dollar FINI grant was awarded to Fair Food Network to
help it pilot three adjustments to the Double Up Food Buck program
\citep{usda_nifa_usda_2015}. First, FFN needs to test DUFB as a
year-round program in select locations instead of the current seasonal
format. Second, shift away from the token system to providing DUFB
electronically at point-of-sale. Third, the DUFB needs to expand from
farmer's markets into other retail environments, like supermarkets and
grocery stores.

Successful expansion into supermarkets and grocery stores is critical.
Approximately 80\% of all SNAP benefits in 2015 were used in
supermarkets or super stores \citep{usda_fns_snap_2016}. Less than 1\%
percent of SNAP benefits were used at local farmer's markets. The amount
of SNAP benefits used in local farmer's markets has increased since
2009, but no where near the growth necessary to reach the type of stores
most frequented by low-income families. If localized financial incentive
programs like DUFB are going to be considered one of the USDA's many
tools to increase food access and combat obesity, then they must be
successfully implemented and scaled across supermarkets and grocery
stores. Most importantly, incentive programs like DUFB must prove they
are effective in changing purchasing habits within supermarket/grocery
store food environments.

\subsection*{Double Up Food Bucks vs the Healthy Incentives
Pilot}\label{double-up-food-bucks-vs-the-healthy-incentives-pilot}
\addcontentsline{toc}{subsection}{Double Up Food Bucks vs the Healthy
Incentives Pilot}

There are notable differences between DUFB and HIP that make the
evaluation of DUFB more difficult. In short, HIP was implemented as an
RCT. DUFB implementation is not. Let's explore in greater detail.

HIP had substantially more participating stores, all within the same
county (Hampden County, MA). DUFB has fewer participating stores, spread
across many different counties, and across many different grocery store
chains. Therefore, the probability of a SNAP shopper in Hampden County
having walked into a HIP participating store was much higher than a SNAP
shopper walking into any DUFB participating retailer.

The incentive delivery mechanisms also differ. First, all SNAP
beneficiaries who shop at a DUFB participating store receive the benefit
automatically. In other words, SNAP households that patron a store with
DUFB receive the incentive regardless of their intentions or awareness
of the DUFB incentive. Therefore, evaluating DUFB has the added
difficulty of identifying which shoppers are optimizing in response to
DUFB, as opposed to shopping normally. In contrast, SNAP households
assigned to the HIP treatment group were made aware of incentive and
were eligible to use it (even if they didn't quite understand how the
incentive program worked --- see \citet{bartlett_evaluation_2014}).
Households in the control group were not aware of the incentive and were
not eligible to use it. And because participants were assigned, HIP
evaluators could identify treated participants from control
participants.

Second, the DUFB financial incentive is substantially larger but more
restrictive. The DUFB incentive is a dollar-for-dollar match of locally
grown produce purchases capped at \$20 per day. The matched dollars are
accrued as points on a store loyalty card. Existing points are then
automatically redeemed as dollars on \emph{any} fresh produce purchases,
not just locally grown produce. In comparison, the HIP financial
incentive was a return of 30 cents per dollar spent on TFVs which could
be spent on \emph{any} food item. That is, the DUFB incentive doubled
the purchasing power of every dollars spent on TFVs \emph{only for more
TFVs}; the HIP incentive increased the purchasing power of every dollars
spent on TFVs by 30\% \emph{for any SNAP eligible food item}.

Finally, the experimental designed of HIP allowed researchers to form a
causal interpretation of their results; the average treatment effect is
the same as the average treatment effect on the treated. Any difference
in the purchase and consumption of TFV between the treatment and control
groups could therefore be attributed to the incentive. This is not the
case for DUFB. However, HIP implementation is the exception. How DUFB,
and similar financial incentive programs are implemented, is the norm.
The contribution of this paper will be evaluating and understanding the
impact of DUFB, given that DUFB and similar programs are implemented in
the ``real-world'' (non-experimental conditions)

\subsection*{Evaluating Double Up Food Bucks in Non-experimental
Conditions}\label{evaluating-double-up-food-bucks-in-non-experimental-conditions}
\addcontentsline{toc}{subsection}{Evaluating Double Up Food Bucks in
Non-experimental Conditions}

DUFB's expansion and implementation into supermarkets and grocery stores
did not follow an experimental design. Fair Food Network searched for
local partners in the Detroit area willing to participate in DUFB. Not
all grocery stores, especially the smaller independent stores, had the
capacity to implement the point-of-sale technology necessary for the
incentive---even if FFN offered to help cover the upgrade costs. The
result is a self-selected group of stores participating in DUFB. This,
in some ways, parallels what occurred in HIP, where one of the largest
retailers decided integrating their point-of-sale systems to include the
incentive was too expensive. This type of strategic firm behavior is
important to consider, even if complicates the evaluation of an
incentive program like DUFB.

In the real world, stores seek to maximize profits and will opt to
participate only if they expect to profit. Similarly, individuals will
self-select into participating; participation is optional and more
likely to occur with well-informed and motivated SNAP shoppers.
Selection, in this case, is a feature, not a flaw, of such incentive
programs when implemented by non-profits or policy makers. The evidence,
thanks to HIP, exists that incentives can lead to an increase in
consumption. The goal of this paper is therefore to accurately measure
the effect of the DUFB on TFV purchases while taking the selection into
account. That effect can then be extrapolated forward, albeit weakly,
using the results of HIP, to measure changes in consumption.

Fair Food Network started testing and gathering data from grocery stores
implementing DUFB in 2014. One of FFN's largest partners, a Michigan
grocery retail and distribution company, piloted the program in 2 of its
stores in 2014. The company has since expanded to 5 stores in 2015 and
then to 17 of 62 stores in 2016. Rapid scaling was possible due to the
point-of-sale technology used by the company to implement DUFB across
its stores. It provides, to date, the best case study of a firm
strategically scaling DUFB across numerous grocery stores that span
different geographic areas and populations.

All transaction data from 2014 - 2016 will be provided for every store
that has, at any point, participated in DUFB. These data are complete
(i.e.~no records have been removed) and at the item level. A complete
set of data will also be provided from another 15 stores where DUFB was
not implemented.

Currently, no research exists evaluating DUFB, or similar incentive
programs, using a complete set of store transaction data. HIP, for
example, only had transactions records for SNAP EBT cards. Transactions,
should a different tender be used by the same individual, could not be
observed. Therefore, these data provide an unprecedented opportunity to
analyze how the DUFB financial incentive performs under real-world
conditions. This paper will be, to the best of my knowledge, the first
to perform an evaluation of a financial incentive, targeted at SNAP
participants, using a complete set of data, from multiple stores, across
multiple years, and collected under non-experimental conditions.

\section*{Data}\label{data-1}
\addcontentsline{toc}{section}{Data}

These data come from a large grocery distributor and retailer serving
multiple grocery chains. Three years of data will be made available,
2014 through 2016. These data are transaction level and will include (at
least) store number, register, transaction ID, date and time of
purchase, payment type, item, dollars, and quantity.

Double Up implementation was considered for a single grocery chain. The
chain has more than 60 stores, 17 of which were selected as
``treatment'' stores (with Double Up). Of the remaining stores, data is
being made available from an addition 15 to serve as ``controls''. The
quotes here signify that these terms will be used as shorthand, but the
terminology is somewhat misleading. The use of ``treatment'' and
``control'' could lead one to think store assignment was random. It was
not.

{[}MISSING real specific details about the data e.g.~total transactions
observed etc.{]}

One important variable that will not be made available is a variable for
loyalty card numbers. The company's use of loyalty cards across its many
chains was an exciting prospect. Previous transaction data from smaller
independent grocery chains had no way linking purchases to a single
unique identifier over time because these smaller chains did not have
advanced point-of-sale systems.

In earlier conversations with the company, it was understood that
loyalty cards would be made available. However, months into working with
the company, we were informed that this was no longer possible. Per the
company's legal department, the company cannot share any personal
information about their customers. Unfortunately for us, in the loyalty
card contract signed by customers, the loyalty card number itself is
considered personal information, meaning loyalty card numbers fall under
the same legal category as phone numbers and home addresses.

\subsection*{Overview of Store Selection and
Expansion}\label{overview-of-store-selection-and-expansion}
\addcontentsline{toc}{subsection}{Overview of Store Selection and
Expansion}

How the 17 ``treatment'' stores and 15 ``control'' stores were selected
in 2016 is important. First and foremost, selection was \emph{not}
random. Stores were either selected by the company (13 of 17) or
self-selected into Double Up (4 of 17). Second, the 15 control stores
were selected \emph{after} the selection of the 17 treatment stores.
Data from all remaining stores was requested but the request was denied;
only 15 stores had been approved by the company's management. Finally,
and most importantly, the selection criteria for the 17 treatment stores
is \emph{observable}. The implications of this will be covered in more
detail in the \protect\hyperlink{methods}{Methods} section.

\subsection*{Selection and Expansion of Double Up
Stores}\label{selection-and-expansion-of-double-up-stores}
\addcontentsline{toc}{subsection}{Selection and Expansion of Double Up
Stores}

The first 2 stores were piloted with Double Up in 2014. Both were in
geographically distinct areas (these will be referred to as
``\texttt{Node\ 0}'' and ``\texttt{Node\ 1}''). There was a small
expansion adding 3 more stores in 2015. The 3 stores were selected
because they were geographically close to the 2 original pilot stores (2
close to \texttt{Node\ 0}, 1 close to \texttt{Node\ 1}). The 5 stores
are referred to as the ``core''. The location of these 5 stores,
separated in two clusters, established the geographic constraints that
were then used to determine most of the additional stores in 2016.

Double Up was expanded to 12 more stores in 2016, totaling 17. Of those
12, 6 were selected due to their proximity to the 5 core stores, their
SNAP EBT\footnote{Electronic Benefit Transfer.} sales figures, and
similarity in surrounding demographics (high population density, more
African-American). In other words, 9 of the 17 stores---excluding the
initial 2 pilot stores-----were selected on a set of \emph{observable}
characteristics. The remaining 6 stores were not.

Of the remaining 6 stores, 4 asked if they could be included in the
program. These stores \emph{self-selected} into Double Up, making these
stores fundamentally distinct. They were considered, and then included,
only because they fell within the ``Top 50''. The final 2 stores were
selected by the company for ``strategic business decision''. The best
interpretation of this is that the company thought that Double Up would
provide a competitive edge to the 2 included stores given some internal
calculus. How the company came to this decision is \emph{unknown} and
therefore \emph{unobserved}.

Table \ref{tab:store-class} helps understand the year by year expansion
of Double Up. Stores are classified as either \texttt{assigned},
\texttt{self-selected}, or \texttt{unobserved}. To be \texttt{assigned}
means a stores participation in Double Up was determined (assigned) by
the company; \texttt{self-selected} means the store asked the company to
participate; \texttt{unobserved} means that the company selected the
store to participate in Double Up but for unknown and unobserved
reasons. Numbers were assigned to each store for easy reference but
otherwise have no meaningful interpretation.

\begin{table}

\caption{\label{tab:store-class}Year by Year Store Selection. Stores 1 and 2 represent the initial 2014 pilot stores.}
\centering
\begin{tabular}[t]{rlll}
\toprule
Store & 2014 & 2015 & 2016\\
\midrule
1 & assigned & assigned & assigned\\
2 & assigned & assigned & assigned\\
3 &  & assigned & assigned\\
4 &  & assigned & assigned\\
5 &  & assigned & assigned\\
\addlinespace
6 &  &  & assigned\\
7 &  &  & assigned\\
8 &  &  & assigned\\
9 &  &  & assigned\\
10 &  &  & assigned\\
\addlinespace
11 &  &  & assigned\\
12 &  &  & self-selected\\
13 &  &  & self-selected\\
14 &  &  & self-selected\\
15 &  &  & self-selected\\
\addlinespace
16 &  &  & unobserved\\
17 &  &  & unobserved\\
\bottomrule
\end{tabular}
\end{table}

\subsection*{Expansion on Observables}\label{expansion-on-observables}
\addcontentsline{toc}{subsection}{Expansion on Observables}

An example expansion on \emph{observables} (using fake data) can be seen
in Figure \ref{fig:dufb-expansion}. In the top frame, one can see two
blue dots. These blue dots simulate the first two pilot stores in 2014.
The left blue dot is \texttt{Node\ 0} and the right blue dot is
\texttt{Node\ 1}. The gray zones represent areas of higher population
density. Dark gray is considered \emph{urban}, defined as having a
population density of 1500 persons or more per square mile. The light
gray are small towns and cities, more densely populated than very rural
areas, but could not be considered \emph{urban}. The expansion in 2015
(middle frame) proceeds to the stores closest to the original pilot
stores. The expansion continues to 6 more stores in 2016 (bottom frame)
away from the nodes but also along areas of higher population density.

Not conveyed in Figure \ref{fig:dufb-expansion} is that the 2015 and
2016 expansions also move through stores that happen to be ``highly
ranked''---that is, have relatively higher SNAP EBT sales.\footnote{All
  stores within the chain were ranked by SNAP EBT sales as a percentage
  of total sales.} Also not conveyed is the fact that there is a strong
correlation between geography, population density, racial composition,
and SNAP EBT sales. The 2015 expansion to the most nearby stores also
meant that it was an expansion to stores with high SNAP EBT sales in
densely populated, African-American neighborhoods. The 2016 Double Up
expansion was more explicit given that set of feasible stores
substantially increases as one moves away from each node. Double Up
stores were thus specifically selected not just by geographic proximity,
but also by SNAP EBT sales ranking and demographic compositions similar
to the initial 2014 stores.

\begin{figure}

{\centering \includegraphics{figures/expansion-v} 

}

\caption{Example expansion over time from 2014 to 2016 (top to bottom) using fake data. Blue dots denote stores with Double Up, pink dots denote without. Gray sectors denote higher population density. The initial nodes can be seen in the top (2014) frame.}\label{fig:dufb-expansion}
\end{figure}

\subsection*{Selection of Control
Stores}\label{selection-of-control-stores}
\addcontentsline{toc}{subsection}{Selection of Control Stores}

Ideally, all remaining stores would have been available to use as a
control group but the company only approved that data be released for 15
stores. This left the added---and incredibly important---step of
selecting the control stores since the company approved, but did not
explicitly select, the 15 stores.

Selecting the control stores proceeded in two steps. First, stores that
either self-selected or were selected using some unobservable criteria
were matched using \emph{Coarsened Exact Matching} (CEM)
\citep{iacus_causal_2011}. Second, stores assigned Double Up were pooled
with nearby control stores and then scored using a linear probability
model. Each step is explained in detail.

\subsubsection{Step 1: Coarsened Exact
Matching}\label{step-1-coarsened-exact-matching}
\addcontentsline{toc}{subsubsection}{Step 1: Coarsened Exact Matching}

The 6 stores classified as \texttt{self-selected} or \texttt{unobserved}
(stores \texttt{12} through \texttt{17}; see Table
\ref{tab:store-class}) were compared against all possible control stores
for matches. Matching was done across 5 dimensions: race, income,
population density, store attributes, store EBT sales. One variable per
dimension was selected: percentage of population that is
African-American (zip code level); people per square mile (zip code
level); median income for people who have received SNAP or similar
assistance (zip code level); the number of associates employed in each
store; and the percentage of total stores sales attributed to EBT/SNAP.

Of the 6 stores (stores \texttt{12} - \texttt{17}), only 3 produced
viable matches. However, each of the 3 matched stores had matched to
more than one control stores. The closest stores, by driving distance,
were selected as the tie-breaker for each matched store. Stores were
sufficiently far apart, with very sparsely populated areas between, that
``spill-over'' was considered unlikely. That is, it is considered
unlikely that a shopper near a store without Double Up would opt to
drive 30 or more minutes to shop at the store \emph{with} Double Up.

This left 12 stores to be allotted to the control group and 3 treatment
stores to be effectively discarded.

\subsubsection{Step 2: Scoring via Linear Probability
Model}\label{step-2-scoring-via-linear-probability-model}
\addcontentsline{toc}{subsubsection}{Step 2: Scoring via Linear
Probability Model}

Assignment to treatment and control can be perfectly determined since we
know and observe the criteria used for assignment: geographic distance
from an initial store (node), SNAP EBT sales rank, and
demographics---specifically population density and percentage
African-American. A scoring function was created by fitting a linear
probability model to all stores within 140 kilometers of the two initial
pilot stores.

\[
\begin{aligned}
  \bm{s}  &= \widehat{P(\mathbf{D} = 1 | \bm{X}, \bm{N})} \\
          &= \mathbf{X} \bm{\hat \beta} + \hat \alpha \mathbf{N} + \left (\mathbf{X} \odot \mathbf{N} \right ) \bm{\hat \gamma}
\end{aligned}
\]

\(\bm{s}\) are the fitted values of the estimated linear probability
model; \(\mathbf{D} \in \{0,1 \}\) is a \(n \times 1\) vector of store
assignments to Double Up; \(\mathbf{X}\) is an \(n \times k\) matrix of
normalized observable covariates that determine assignment;
\(\mathbf{N} \in \{0, 1 \}\) is an \(n \times 1\) dummy vector denoting
the closest pilot store aka ``Node'', where \(0\) is \texttt{Node\ 0}
and \(1\) is \texttt{Node\ 1}. \(\odot\) represents element-wise
multiplication aka ``Hadamard product''.

Stores were sorted by the fitted values of the model, \(\bm{s}\). There
is perfect separation between Double Up stores and those without (see
Figure \ref{fig:score-plot}). Therefore, the top 11 stores by score
value are all Double Up stores. The next 12 stores by score value are
then allotted to the control group.

\begin{figure}
\centering
\includegraphics{noriega-prospectus-draft_files/figure-latex/score-plot-1.pdf}
\caption{\label{fig:score-plot}Store Score vs Double Up Assignment}
\end{figure}

\bibliography{bib/book.bib}

\backmatter
\printindex

\end{document}
