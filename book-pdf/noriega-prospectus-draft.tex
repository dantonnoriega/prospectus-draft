\documentclass[12pt,letterpaperpaper,]{book}
\usepackage{lmodern}
\usepackage{amssymb,amsmath}
\usepackage{ifxetex,ifluatex}
\usepackage{fixltx2e} % provides \textsubscript
\ifnum 0\ifxetex 1\fi\ifluatex 1\fi=0 % if pdftex
  \usepackage[T1]{fontenc}
  \usepackage[utf8]{inputenc}
\else % if luatex or xelatex
  \ifxetex
    \usepackage{mathspec}
  \else
    \usepackage{fontspec}
  \fi
  \defaultfontfeatures{Ligatures=TeX,Scale=MatchLowercase}
\fi
% use upquote if available, for straight quotes in verbatim environments
\IfFileExists{upquote.sty}{\usepackage{upquote}}{}
% use microtype if available
\IfFileExists{microtype.sty}{%
\usepackage{microtype}
\UseMicrotypeSet[protrusion]{basicmath} % disable protrusion for tt fonts
}{}
\usepackage[margin=1.2in]{geometry}
\usepackage{hyperref}
\hypersetup{unicode=true,
            pdftitle={Dissertation Prospectus},
            pdfauthor={Danton Noriega},
            pdfborder={0 0 0},
            breaklinks=true}
\urlstyle{same}  % don't use monospace font for urls
\usepackage{natbib}
\bibliographystyle{apalike}
\usepackage{longtable,booktabs}
\usepackage{graphicx,grffile}
\makeatletter
\def\maxwidth{\ifdim\Gin@nat@width>\linewidth\linewidth\else\Gin@nat@width\fi}
\def\maxheight{\ifdim\Gin@nat@height>\textheight\textheight\else\Gin@nat@height\fi}
\makeatother
% Scale images if necessary, so that they will not overflow the page
% margins by default, and it is still possible to overwrite the defaults
% using explicit options in \includegraphics[width, height, ...]{}
\setkeys{Gin}{width=\maxwidth,height=\maxheight,keepaspectratio}
\IfFileExists{parskip.sty}{%
\usepackage{parskip}
}{% else
\setlength{\parindent}{0pt}
\setlength{\parskip}{6pt plus 2pt minus 1pt}
}
\setlength{\emergencystretch}{3em}  % prevent overfull lines
\providecommand{\tightlist}{%
  \setlength{\itemsep}{0pt}\setlength{\parskip}{0pt}}
\setcounter{secnumdepth}{5}
% Redefines (sub)paragraphs to behave more like sections
\ifx\paragraph\undefined\else
\let\oldparagraph\paragraph
\renewcommand{\paragraph}[1]{\oldparagraph{#1}\mbox{}}
\fi
\ifx\subparagraph\undefined\else
\let\oldsubparagraph\subparagraph
\renewcommand{\subparagraph}[1]{\oldsubparagraph{#1}\mbox{}}
\fi
\usepackage{booktabs}
\usepackage{lscape}
\usepackage{longtable}
\usepackage{bm}
\usepackage{isomath}
\usepackage[bf,singlelinecheck=off]{caption}
\newcommand{\matr}[1]{\mathbf{#1}}
\newcommand{\blandscape}{\begin{landscape}}
\newcommand{\elandscape}{\end{landscape}}

\setromanfont[Mapping=tex-text]{Minion Pro}
\setsansfont[Mapping=tex-text]{Source Sans Pro}
\setmonofont[Mapping=tex-text,Scale=0.8]{Source Code Pro}

\usepackage{framed,color}
\definecolor{shadecolor}{RGB}{248,248,248}

\renewcommand{\textfraction}{0.05}
\renewcommand{\topfraction}{0.8}
\renewcommand{\bottomfraction}{0.8}
\renewcommand{\floatpagefraction}{0.75}

\renewenvironment{quote}{\begin{VF}}{\end{VF}}
\let\oldhref\href
\renewcommand{\href}[2]{#2\footnote{\url{#1}}}

\ifxetex
  \usepackage{letltxmacro}
  \setlength{\XeTeXLinkMargin}{1pt}
  \LetLtxMacro\SavedIncludeGraphics\includegraphics
  \def\includegraphics#1#{% #1 catches optional stuff (star/opt. arg.)
    \IncludeGraphicsAux{#1}%
  }%
  \newcommand*{\IncludeGraphicsAux}[2]{%
    \XeTeXLinkBox{%
      \SavedIncludeGraphics#1{#2}%
    }%
  }%
\fi

\makeatletter
\newenvironment{kframe}{%
\medskip{}
\setlength{\fboxsep}{.8em}
 \def\at@end@of@kframe{}%
 \ifinner\ifhmode%
  \def\at@end@of@kframe{\end{minipage}}%
  \begin{minipage}{\columnwidth}%
 \fi\fi%
 \def\FrameCommand##1{\hskip\@totalleftmargin \hskip-\fboxsep
 \colorbox{shadecolor}{##1}\hskip-\fboxsep
     % There is no \\@totalrightmargin, so:
     \hskip-\linewidth \hskip-\@totalleftmargin \hskip\columnwidth}%
 \MakeFramed {\advance\hsize-\width
   \@totalleftmargin\z@ \linewidth\hsize
   \@setminipage}}%
 {\par\unskip\endMakeFramed%
 \at@end@of@kframe}
\makeatother

\newenvironment{rmdblock}[1]
  {
  \begin{itemize}
  \renewcommand{\labelitemi}{
    \raisebox{-.7\height}[0pt][0pt]{
      {\setkeys{Gin}{width=3em,keepaspectratio}\includegraphics{images/#1}}
    }
  }
  \setlength{\fboxsep}{1em}
  \begin{kframe}
  \item
  }
  {
  \end{kframe}
  \end{itemize}
  }
\newenvironment{rmdnote}
  {\begin{rmdblock}{note}}
  {\end{rmdblock}}
\newenvironment{rmdcaution}
  {\begin{rmdblock}{caution}}
  {\end{rmdblock}}
\newenvironment{rmdimportant}
  {\begin{rmdblock}{important}}
  {\end{rmdblock}}
\newenvironment{rmdtip}
  {\begin{rmdblock}{tip}}
  {\end{rmdblock}}
\newenvironment{rmdwarning}
  {\begin{rmdblock}{warning}}
  {\end{rmdblock}}

\usepackage{makeidx}
\makeindex

\urlstyle{tt}

\usepackage{amsthm}
\makeatletter
\def\thm@space@setup{%
  \thm@preskip=8pt plus 2pt minus 4pt
  \thm@postskip=\thm@preskip
}
\makeatother

\title{Dissertation Prospectus}
\author{Danton Noriega}
\date{November 11 2016}

\begin{document}
\maketitle

\setlength{\abovedisplayskip}{-5pt}
\setlength{\abovedisplayshortskip}{-5pt}
\frontmatter

{
\setcounter{tocdepth}{2}
\tableofcontents
}
\chapter{Overview}\label{overview}

Chapter 1 is an evaluation of the effectiveness of the Double Up Food
Bucks (Double Up) program. ``Effectiveness'' will be defined by the
change in total sales and volume of produce sold within a subset of
grocery stores that implement Double Up (treatment group). The control
group comprises 15 stores where Double Up was not implemented. A
difference-in-difference between stores using Double Up (treatment) and
those without (control) will be used to measure the size of the effect.

The broader policy concern is that of of improving health and food
equity of SNAP participants through targeted fruit and vegetables
subsidies. The focal point will be on the
\href{http://www.doubleupfoodbucks.org/}{\emph{Double Up Food Bucks}}
program run by the non-profit \emph{Fair Food Network} based out of
Michigan. A comparison will be made with another subsidy program called
the \emph{Healthy Incentives Pilot} (HIP). I will argue how and why my
results from the Double Up program are more realistic and dependable
than those from HIP. However, I will remain neutral on whether
subsidizing fruit and vegetable purchases lead to improved health
outcomes.

Chapter 2 is an exploration of how the basket of products purchased by
SNAP shoppers changes after the introduction of DUFB. This will use a
few different sets of data where one is able to link purchases to
individuals.

Chapter 3 is a descriptive paper on how Durham Social Services spends
TANF dollars. There is no uniformity between how different states and
local governments use TANF funds. There is also little published data
research.

\chapter{An Evaluation of the Double Up Food Bucks
program}\label{chapter-1}

\section*{Introduction}\label{intro-1}
\addcontentsline{toc}{section}{Introduction}

\textbf{Part 1}

Unhealthy eating is expensive. Chronic diet-related conditions like
obesity, heart disease, and other metabolic risk factors (stroke, type
II diabetes, etc) are estimated to cost the US health care system
between 200 to 400 billion dollars annually
\citep{cawley_medical_2012, chatterjee_checkup_2014}. More importantly,
diseases linked to poor diet account for hundreds of thousands of death
each year. Heart disease alone is the leading cause of death for all
persons in the US, with stroke fifth and diabetes seventh
\citep{national_center_for_health_statistics_health_2015}. Improving the
diet of Americans has therefore become an increasing priority for the
United States, especially for struggling families.

Obesity and heart disease rates vary by income, affecting more
low-income families than middle- and high-income families. Poverty and
poor health outcomes, in turn, correlate with food insecurity and poor
diet. Research on the dietary patterns of households receiving
Supplemental Nutrition Assistance (SNAP) benefits has found that they
are significantly \emph{less} likely to meet USDA dietary guidelines
than the average US household and much \emph{more} likely to consume
unhealthy foods
\citep{andreyeva_dietary_2015, nguyen_supplemental_2015, wolfson_fruit_2015}.
A smaller set of research has found that SNAP households, at best,
consume same amount of unhealthy foods (e.g.~sugar-sweetened beverages,
baked goods, snacks, candy, etc) compared to non-households and
ineligible households \citep{todd_caloric_2014, hoynes_snap_2014}. In
other words, SNAP households consume foods that are less healthy or
about the same as ineligible SNAP households. This is a concerning
result given that most US households already purchase and consume far
too much meat and foods rich in sugars and fats, and far too few fruits,
vegetables and whole grains
\citep{usda_scientific_2015, frazao_high_1999}.

However, SNAP is first and foremost an anti-hunger program, not health
and nutrition program. To qualify for SNAP, a household must be
sufficiently budget constrained that \textbf{hunger} is considered
likely without cash assistance. Eligibility, is therefore, only a
function of income.

There has been extensive research on the role that SNAP does, can, or
should play in helping improve the health of a struggling families. As a
consequence, SNAP beneficiaries can often not afford the luxury of
substituting healthy foods for unhealthy foods when unhealthy foods can
be cheaper, more accessible, and take less time to prepare. It is a
costly trade-off between hunger and health.

\textbf{Note}: I need to find a better way to include
\citet{blumenthal_strategies_2014}; \citet{leung_qualitative_2013};
\citet{cawley_economy_2015}. All combined, they provide a solid review
of what experts see as barriers for healthy food for SNAP beneficiaries
as well as the recommendations to improve their diets.

\textbar{}\textbar{}But SNAP is also federal aid program. In response to
seeing federal dollars being spent on unhealthy foods, some policy
makers have suggested making SNAP more like the Woman, Infants, and
Children (WIC) program \citep{scott_wisconsin_2013}. The WIC program
provides food vouchers which limit households to a select group of
products. These products are specifically selected to be healthy to
ensure women and their children receive nutritious, healthy foods. The
use of WIC benefits, by design, places restrictions on food choices.
This is in contrast to SNAP where there are few restrictions and
households are free to purchase cookies, chips, soda, and any other
assortment of processed foods that have no WIC voucher
equivalent.\textbar{}\textbar{}

For many SNAP beneficiaries, freedom of product choice is what makes the
SNAP program popular and easy to use
\citep{lindsey_wic_2013, edin_snap_2013}. Rather than change SNAP and
make it more restrictive like WIC, the USDA's Food and Nutrition
Services (FNS) has instead renewed its efforts to combat obesity by
providing educational materials and ``toolkits'' to states
\citep{multiple_supplemental_2017}. But of greater interest is how the
USDA is also beginning to test out financial incentives as a way to
improve the diet of SNAP beneficiaries.

\subsection*{Financial Incentives to Encourage Healthy Food
Purchases}\label{financial-incentives-to-encourage-healthy-food-purchases}
\addcontentsline{toc}{subsection}{Financial Incentives to Encourage
Healthy Food Purchases}

Section 4208.(b) of the Agricultural Act of 2014 established the Food
Insecurity Nutrition Initiative (FINI). The allocation of \$100 million
dollars to FINI suggests that the federal government has started to
recognize that ineffectiveness of past policies to change the purchasing
habits of SNAP participating households. Moving forward, the federal and
state governments may play a more active role to increasing the quality
and healthiness of foods accessible to SNAP households. For the moment,
the federal government appears unwilling to change purchasing patterns
by restricting food choice. Instead, the government is opting to pilot
incentives-based programs.

The incentive programs piloted by FINI grants aim to make it easier and
more affordable for SNAP households to purchase (and hopefully consume)
healthier foods without restricting choice. However, not enough evidence
exists yet to know whether, in practice, incentives-based programs are
more effective than restriction-based ones---or if effective at all.
This is why all programs funded through FINI grants must agree to be
properly evaluated.

Of specific interest is Double Up Food Bucks (Double Up), an
incentives-based program funded by FINI. Double Up, launched in 2009 by
the non-profit organization Fair Food Networks (FFN), doubles the
purchasing power of SNAP recipients buying produce. Dollars spent on
Michigan produce are match up to \$20 dollars, but the matching funds
can only be used to purchase more fruits and vegetables at a later date.
The program is considered by FFN to be a ``three-fold'' win: it helps
local low-income families buy more fresh produce, provides new customers
for local farmers, and stimulates the local food economy
\citep{fair_food_network_double_2014}.

\subsection*{The Double Up Food Bucks
Program}\label{the-double-up-food-bucks-program}
\addcontentsline{toc}{subsection}{The Double Up Food Bucks Program}

The non-profit organization Fair Food Networks (FFN) launched the Double
Up Food Bucks (Double Up) program in 2009 in Detroit, Michigan. The
intention of the program was to get more low-income families visiting
and participating in local Detroit farmers markets. The mechanism for
increasing participation was a financial incentive: a dollar-for-dollar
match for every for fruits and vegetables. This subsidy was accessible
only to low-income families receiving SNAP benefits, who could exchange
up to \$20 of their benefits for a wooden token that could be used on up
to \$40 worth of locally grown, fresh produce.

The Double Up program was considered successful given it had expanded to
more than 150 farmers markets in 2014 from just 5 farmers markets in
2009. SNAP benefits have been used more than 200,000 times to purchase
fresh produce, with more than 10,000 first time SNAP customers visiting
farmers markets in 2013 alone \citep{fair_food_network_double_2014}. The
program is considered by Fair Food Network to be a ``three-fold'' win
given that the program helps local low-income families buy more fresh
produce, provides new customers for local farmers, and stimulates the
local food economy. Relative to farmers markets in other states, Double
Up did seem to be bringing in substantially more SNAP dollars (\$1.7
million in Michigan versus \$307,000 in Illinois, the second largest).

\subsection*{From Farmers Markets to Grocery
Stores}\label{from-farmers-markets-to-grocery-stores}
\addcontentsline{toc}{subsection}{From Farmers Markets to Grocery
Stores}

By successfully expanding the program over 5 years into many different
communities, FFN established itself as a consistent and reliable partner
in numerous local farmers markets across many diverse communities. In
short, it proved that Double Up was scalable. The problem, however, is
that while FFN was able to prove they could successfully scale Double Up
across other farmers markets, FFN had not yet proven they could expand
Double Up into local supermarkets and grocery stores.

A 5.17 million dollar FINI grant was awarded to Fair Food Network to
help it pilot three adjustments to the Double Up Food Buck program
\citep{usda_nifa_usda_2015}. First, FFN needs to test Double Up as a
year-round program in select locations instead of the current seasonal
format. Second, shift away from the token system to the electronic
processing of Double Up transactions. Third, the Double Up needs to
expand from farmers markets to retail supermarkets and grocery stores.

Successful expansion into supermarkets and grocery stores is critical.
Approximately three-quarters of all SNAP benefits in 2009 were used in
supermarkets, super-centers, or small to large grocery stores
\citep{castner_benefit_2011}. Less than 7\% percent of SNAP benefits
were used at local farmers markets. The amount of SNAP benefits used in
local farmers markets has increased since 2009, but no where near the
growth necessary to reach the type of stores most frequented by
low-income families. If incentive programs like Double Up are going to
be considered as one of the USDA's many tools to increase food access
and combat obesity, then Double Up must be successfully implemented and
scaled across supermarkets and grocery stores. Above all else, Double Up
must prove it is effective in changing purchasing habits within the
supermarket/grocery store food environment.

The Fair Food Network started testing and gathering data from grocery
stores implementing Double Up in 2014. The mechanisms used to implement
Double Up varies across grocery stores and chains, as does the produce
offered and customer demographics. One of FFN's partners largest
partners, a Michigan grocery retail and distribution company, piloted
the program in 2 of its stores in 2014. The company has since expanded
to 5 stores in 2015 and then to 17 of 62 stores in 2016. Rapid scaling
was possible due to the point-of-sale technology used by the company to
implement Double Ups across its stores. It provides, to date, the best
case study of scaling Double Up across numerous grocery stores that span
different geographic areas and populations for a specific incentive
mechanism. Data from 2014 - 2016 will also be provided from another 15
stores where Double Up was not implement.

The availability of these data will provide an unprecedented look into
how SNAP shoppers respond to targeted financial incentives. No research
has been done on food incentives specifically tailored towards SNAP
shoppers where behavior was analyzed via transactions as opposed to
surveys (see \citep{bartlett_evaluation_2014}). Furthermore, no research
has been done where the financial incentive was this large.

This paper aims to evaluate the effectiveness of the Double Up Food
Bucks.

\section{Data}\label{data-1}

These data come from a large grocery distributor and retailer serving
multiple grocery chains. Three years of data will be made available,
2014 through 2016. These data are transaction level and will include (at
least) store number, register, transaction ID, date and time of
purchase, payment type, item, dollars, and quantity.

Double Up implementation was considered for a single grocery chain. The
chain has more than 60 stores, 17 of which were selected as
``treatment'' stores (with Double Up). Of the remaining stores, data is
being made available from an addition 15 to serve as ``controls''. The
quotes here signify that these terms will be used as shorthand, but the
terminology is somewhat misleading. The use of ``treatment'' and
``control'' could lead one to think store assignment was random. It was
not.

{[}TK real specific details about the data e.g.~total transactions
observed etc.{]}

One important variable that will not be made available is a variable for
loyalty card numbers. The company's use of loyalty cards across its many
chains was an exciting prospect. Previous transaction data from smaller
independent grocery chains had no way linking purchases to a single
unique identifier over time because these smaller chains did not have
advanced point-of-sale systems.

In earlier conversations with the company, it was understood that
loyalty cards would be made available. However, months into working with
the company, we were informed that this was no longer possible.
According to the company's legal department, the company cannot share
any personal information about their customers. Unfortunately for us, in
the loyalty card contract signed by customers, the loyalty card number
itself is considered personal information, meaning loyalty card numbers
fall under the same legal category as phone numbers and home addresses.

\subsection*{Overview of Store Selection and
Expansion}\label{overview-of-store-selection-and-expansion}
\addcontentsline{toc}{subsection}{Overview of Store Selection and
Expansion}

How the 17 ``treatment'' stores and 15 ``control'' stores were selected
in 2016 is important. First and foremost, selection was \emph{not}
random. Stores were either selected by the company (13 of 17) or
self-selected into Double Up (4 of 17). Second, the 15 control stores
were selected \emph{after} the selection of the 17 treatment stores.
Data from all remaining stores was requested but the request was denied;
only 15 stores had been approved by the company's management. Finally,
and most importantly, the selection criteria for the 17 treatment stores
is \emph{observable}. The implications of this will be covered in more
detail in the \protect\hyperlink{methods}{Methods} section.

\subsection*{Selection and Expansion of Double Up
Stores}\label{selection-and-expansion-of-double-up-stores}
\addcontentsline{toc}{subsection}{Selection and Expansion of Double Up
Stores}

The first 2 stores were piloted with Double Up in 2014. Both were in
geographically distinct areas (these will be referred to as
``\texttt{Node\ 0}'' and ``\texttt{Node\ 1}''). There was a small
expansion adding 3 more stores in 2015. The 3 stores were selected
because they were geographically close to the 2 original pilot stores (2
close to \texttt{Node\ 0}, 1 close to \texttt{Node\ 1}). The 5 stores
are referred to as the ``core''. These location of these 5 stores,
separated in two clusters, established the geographic constraints that
were then used to determine most of the additional stores in 2016.

Double Up was expanded to 12 more stores in 2016, totaling 17. Of those
12, 6 were selected due to their proximity to the 5 core stores, their
SNAP EBT\footnote{Electronic Benefit Transfer.} sales figures, and
similarity in surrounding demographics (high population density, more
African-American). In other words, 9 of the 17 stores---excluding the
initial 2 pilot stores-----were selected on a set of \emph{observable}
characteristics. The remaining 6 stores were not.

Of the remaining 6 stores, 4 asked if they could be included in the
program. These stores \emph{self-selected} into Double Up, making these
stores fundamentally distinct. They were considered, and then included,
only because they fell within the ``Top 50''. The final 2 stores were
selected by the company for ``strategic business decision''. The best
interpretation of this is that the company thought that Double Up would
provide a competitive edge to the 2 included stores given some internal
calculus. How the company came to this decision is \emph{unknown} and
therefore \emph{unobserved}.

Table \ref{tab:store-class} helps understand the year by year expansion
of Double Up. Stores are classified as either \texttt{assigned},
\texttt{self-selected}, or \texttt{unobserved}. To be \texttt{assigned}
means a stores participation in Double Up was determined (assigned) by
the company; \texttt{self-selected} means the store asked the company to
participate; \texttt{unobserved} means that the company selected the
store to participate in Double Up but for unknown and unobserved
reasons. Capital letters (i.e.~A, B, C) were assigned to each store for
easy reference but otherwise have no meaningful interpretation.

\begin{table}

\caption{\label{tab:store-class}Year by Year Store Selection. Stores 1 and 2 represent the initial 2014 pilot stores.}
\centering
\begin{tabular}[t]{rlll}
\toprule
Store & 2014 & 2015 & 2016\\
\midrule
1 & assigned & assigned & assigned\\
2 & assigned & assigned & assigned\\
3 &  & assigned & assigned\\
4 &  & assigned & assigned\\
5 &  & assigned & assigned\\
\addlinespace
6 &  &  & assigned\\
7 &  &  & assigned\\
8 &  &  & assigned\\
9 &  &  & assigned\\
10 &  &  & assigned\\
\addlinespace
11 &  &  & assigned\\
12 &  &  & self-selected\\
13 &  &  & self-selected\\
14 &  &  & self-selected\\
15 &  &  & self-selected\\
\addlinespace
16 &  &  & unobserved\\
17 &  &  & unobserved\\
\bottomrule
\end{tabular}
\end{table}

\subsection*{Expansion on Observables}\label{expansion-on-observables}
\addcontentsline{toc}{subsection}{Expansion on Observables}

An example expansion on \emph{observables} (using fake data) can be seen
in Figure \ref{fig:dufb-expansion}. In the top frame, one can see two
blue dots. These blue dots simulate the first two pilot stores in 2014.
The left blue dot is \texttt{Node\ 0} and the right blue dot is
\texttt{Node\ 1}. The gray zones represents areas of higher population
density. Dark gray is considered \emph{urban}, defined as having a
population density of 1500 persons or more per square mile. The light
gray are small towns and cities, more densely populated than very rural
areas, but could not be considered \emph{urban}. The expansion in 2015
(middle frame) proceeds to the stores closest to the original pilot
stores. The expansion continues to 6 more stores in 2016 (bottom frame)
away from the nodes but also along areas of higher population density.

Not conveyed in Figure \ref{fig:dufb-expansion} is that the 2015 and
2016 expansions also move through stores that happen to be ``highly
ranked''---that is, have relatively higher SNAP EBT sales.\footnote{All
  stores within the chain were ranked by SNAP EBT sales as a percentage
  of total sales.} Also not conveyed is the fact that there is a strong
correlation between geography, population density, racial composition,
and SNAP EBT sales. The 2015 expansion to the most nearby stores also
meant that it was an expansion to stores with high SNAP EBT sales in
densely populated, African-American neighborhoods. The 2016 Double Up
expansion was more explicit given that set of feasible stores
substantially increases as one moves away from each node. Double Up
stores were thus specifically selected not just by geographic proximity,
but also by SNAP EBT sales ranking and demographic compositions similar
to the initial 2014 stores.

\begin{figure}

{\centering \includegraphics{figures/expansion-v} 

}

\caption{Example expansion over time from 2014 to 2016 (top to bottom) using fake data. Blue dots denote stores with Double Up, pink dots denote without. Gray sectors denote higher population density. The initial nodes can be seen in the top (2014) frame.}\label{fig:dufb-expansion}
\end{figure}

\subsection*{Selection of Control
Stores}\label{selection-of-control-stores}
\addcontentsline{toc}{subsection}{Selection of Control Stores}

Ideally, all remaining stores would have been available to use as a
control group but the company only approved that data be released for 15
stores. This left the added---and incredibly important---step of
selecting the control stores since the company approved, but did not
explicitly select, the 15 stores.

Selecting the control stores proceeded in two steps. First, stores that
either self-selected or were selected using some unobservable criteria
were matched using \emph{Coarsened Exact Matching} (CEM)
\citep{iacus_causal_2011}. Second, stores assigned Double Up were pooled
with nearby control stores and then scored using a linear probability
model. Each step is explained in detail.

\subsubsection{Step 1: Coarsened Exact
Matching}\label{step-1-coarsened-exact-matching}
\addcontentsline{toc}{subsubsection}{Step 1: Coarsened Exact Matching}

The 6 stores classified as \texttt{self-selected} or \texttt{unobserved}
(stores \texttt{12} through \texttt{17}; see Table
\ref{tab:store-class}) were compared against all possible control stores
for matches. Matching was done across 5 dimensions: race, income,
population density, store attributes, store EBT sales. One variable per
dimension was selected: percentage of population that is
African-American (zip code level); people per square mile (zip code
level); median income for people who have received SNAP or similar
assistance (zip code level); the number of associates employed in each
store; and the percentage of total stores sales attributed to EBT/SNAP.

Of the 6 stores (stores \texttt{12} - \texttt{17}), only 3 produced
viable matches. However, each of the 3 matched stores had matched to
more than one control stores. The closest stores, by driving distance,
were selected as the tie-breaker for each matched store. Stores were
sufficiently far apart, with very sparsely populated areas between, that
``spill-over'' was considered unlikely. That is, it is considered
unlikely that a shopper near a store without Double Up would opt to
drive 30 or more minutes to shop at the store \emph{with} Double Up.

This left 12 stores to be allotted to the control group and 3 treatment
stores to be effectively discarded.

\subsubsection{Step 2: Scoring via Linear Probability
Model}\label{step-2-scoring-via-linear-probability-model}
\addcontentsline{toc}{subsubsection}{Step 2: Scoring via Linear
Probability Model}

Assignment to treatment and control can be perfectly determined since we
know and observe the criteria used for assignment: geographic distance
from an initial store (node), SNAP EBT sales rank, and
demographics---specifically population density and percentage
African-American. A scoring function was created by fitting a linear
probability model to all stores within 140 kilometers of the two initial
pilot stores.

\[
\begin{aligned}
  \bm{s}  &= \widehat{P(\mathbf{D} = 1 | \bm{X}, \bm{N})} \\
          &= \mathbf{X} \bm{\hat \beta} + \hat \alpha \mathbf{N} + \left (\mathbf{X} \odot \mathbf{N} \right ) \bm{\hat \gamma}
\end{aligned}
\]

\(\bm{s}\) are the fitted values of the estimated linear probability
model; \(\mathbf{D} \in \{0,1 \}\) is a \(n \times 1\) vector of store
assignments to Double Up; \(\mathbf{X}\) is an \(n \times k\) matrix of
normalized observable covariates that determine assignment;
\(\mathbf{N} \in \{0, 1 \}\) is an \(n \times 1\) dummy vector denoting
the closest pilot store aka ``Node'', where \(0\) is \texttt{Node\ 0}
and \(1\) is \texttt{Node\ 1}. \(\odot\) represents element-wise
multiplication aka ``Hadamard product''.

Stores were sorted by the fitted values of the model, \(\bm{s}\). There
is perfect separation between Double Up stores and those without (see
Figure \ref{fig:score-plot}). Therefore, the top 11 stores by score
value are all Double Up stores. The next 12 stores by score value are
then allotted to the control group.

\begin{figure}
\centering
\includegraphics{noriega-prospectus-draft_files/figure-latex/score-plot-1.pdf}
\caption{\label{fig:score-plot}Store Score vs Double Up Assignment}
\end{figure}

\section{Methods}\label{methods-1}

\subsection*{Part 1: Mixed Methods}\label{part-1-mixed-methods}
\addcontentsline{toc}{subsection}{Part 1: Mixed Methods}

A mixture of methods will be used. This is a consequence of the store
selection issue outline in the data section.

\subsection*{Part 2: Methods for Selection on
Observables}\label{part-2-methods-for-selection-on-observables}
\addcontentsline{toc}{subsection}{Part 2: Methods for Selection on
Observables}

Analysis of stores selected using \emph{known} and \emph{observable}
criteria.

\begin{enumerate}
\def\labelenumi{\arabic{enumi}.}
\setcounter{enumi}{1}
\tightlist
\item
  \textbf{Sample}: 11 Treated stores, 12 Control Stores across 3 years,
  weekly values

  \begin{enumerate}
  \def\labelenumii{\arabic{enumii}.}
  \setcounter{enumii}{2}
  \tightlist
  \item
    Within-group and between-group variation

    \begin{enumerate}
    \def\labelenumiii{\arabic{enumiii}.}
    \setcounter{enumiii}{3}
    \tightlist
    \item
      We lose some within-group variation as the 2 pilot stores are only
      ever observed as treated stores.
    \end{enumerate}
  \item
    Stores enter in waves. The staggering is used to produce extra
    variation between groups.
  \item
    T is 58 x 3 (weeks)

    \begin{enumerate}
    \def\labelenumiii{\arabic{enumiii}.}
    \setcounter{enumiii}{5}
    \tightlist
    \item
      Weeks is important because of MI SNAP schedule
    \end{enumerate}
  \item
    N is the same across time but the staggering shifts N for treated
    and control.

    \begin{enumerate}
    \def\labelenumiii{\arabic{enumiii}.}
    \setcounter{enumiii}{7}
    \tightlist
    \item
      \textbf{Table}: Create table showing the change in N subsample.\\
    \end{enumerate}
  \end{enumerate}
\item
  \textbf{Method 1}: Difference in Differences with FE

  \begin{enumerate}
  \def\labelenumii{\arabic{enumii}.}
  \setcounter{enumii}{4}
  \tightlist
  \item
    DinD seems appropriate.
  \item
    FE should remove any store-specific time-invariant attributes.
  \end{enumerate}
\item
  \textbf{Method 2}: Regression Discontinuity

  \begin{enumerate}
  \def\labelenumii{\arabic{enumii}.}
  \setcounter{enumii}{3}
  \tightlist
  \item
    \textbf{Fear 1:} using a running variable that is a function of
    other variables. Only seen one paper reference anything beyond one
    running variable \citep{papay_extending_2011}. Never seen a running
    variable that is score function.
  \item
    \textbf{Fear 2}: sample size here is small. RD performs better with
    large sample size. \#\#\# Part 3: DinD with matching
  \end{enumerate}
\end{enumerate}

Tiny analysis of the 3 stores that self-selected and were matched to 3
other stores.

\begin{enumerate}
\def\labelenumi{\arabic{enumi}.}
\setcounter{enumi}{2}
\tightlist
\item
  \textbf{Method:} Difference in difference been years 2015 and 2016

  \begin{enumerate}
  \def\labelenumii{\arabic{enumii}.}
  \setcounter{enumii}{2}
  \tightlist
  \item
    Power here will be small. Total N will be 6. Nothing will be
    significant. But need to be transparent.
  \end{enumerate}
\end{enumerate}

\bibliography{bib/book.bib}

\backmatter
\printindex

\end{document}
